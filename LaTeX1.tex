\documentclass[a4paper,12pt]{article}
\usepackage[MeX]{polski}
\usepackage[utf8]{inputenc}

%opening
\title{Programy użytkowe - ćwiczenia 2 }
\author{Dawid Laskowski}

\begin{document}
\maketitle

\section{  Fromuły matematyczne w TeXu}\label{sec:tekst}

Przetrenuj w Texu matematyczbych formuł i symboli z rozdziału 1 poczym wykonaj polecenie z rozdziału 2.

\subsection{  Zapis Matematyczny}\label{sec:zapis}

\subsubsection{  Tryb Matematyczny}\label{sec:tryb}
\frenchspacing
Tryb matematyczny 'inline'-wzory pisane w lini tekstu wstawiamy przy pomocy \$ wzór \$( wzór wpisujemy w pojedyńcze dolary)
\frenchspacing

Ułamek w tekscie
$$
 \frac{1}{x}\\
$$

Oto równanie 
$$
c^{2}=a^{2}+b^{2}
$$
\frenchspacing
Tryb matematyczny z zastosowaniu podwójnych dolarów  \$\$wzór\$\$ 

Można odnieść się do powyższych wzorów wykorzytująć polecenie eqref



\end{document}

\end{abstract}

\section{}

\end{document}